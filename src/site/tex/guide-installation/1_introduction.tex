\chapter{Introduction}

Ce document a pour objectif de guider l'utilisateur dans son installation de la plateforme GeOxygene sous Windows ou sous Linux. On décrira d'abord l'installation des outils nécessaires, puis dans un second temps nous préciserons les démarches d'import et de compilation du code source. 

\bigskip

\noindent
Ce guide s'adresse principalement aux développeurs qui souhaitent faire leurs premiers pas dans GeOxygene.

\bigskip
\noindent
Il est à noter que la procédure d'installation et de configuration expliquée ici se base sur la version 1.5 de GeOxygene.

%-----------------------------------------------------------
\bigskip

\begin{flushleft}
    \bf
    Prérequis
\end{flushleft}

\noindent
GeOxygene est un projet Open Source écrit en JAVA. Pour pouvoir fonctionner, GeOxygene nécessite donc l'installation d'une version 6 (ou supérieure) d'une JDK. Il est possible de télécharger cet environnement sur le site de Sun à l'adresse suivante :

\bigskip

\begin{center}
\href{http://www.oracle.com/technetwork/java/javase/downloads/index.html}{http://www.oracle.com/technetwork/java/javase/downloads/index.html}
\end{center}
