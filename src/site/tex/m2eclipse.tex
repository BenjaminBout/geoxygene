%-----------------------------------------------------------------------
% Installation : Eclipse, m2eclipse et subeclipse
% 
%-----------------------------------------------------------------------
\chapter{Installation des outils}

Le plugin m2eclipse fournit une intégration de Maven dans Eclipse. Il interragit aussi avec le plugin subeclipse afin d'importer des projets depuis des dépôts SVN.


%-----------------------------------------------------------------------
\section{Eclipse}
\subsection{Téléchargement}
Vous pouvez télécharger Eclipse sur le 
\href{http://www.eclipse.org/downloads/}{site de téléchargement d'Eclipse}. A priori, toutes les versions destinées au développement Java devraient fonctionner. Néanmoins, seules les versions Helios(3.6), Indigo(3.7) et Juno (4.2) ont été testées dans le cadre de cette documentation. Le téléchargement de cette dernière version d'Eclipse pour les différentes plate-formes est directement accessible ici~:\\

\begin{center}
\begin{tabular}[!t]{lll}
Windows&
\href{http://www.eclipse.org/downloads/download.php?file=/technology/epp/downloads/release/juno/SR1/eclipse-jee-juno-SR1-win32.zip}{32bit}&
\href{http://www.eclipse.org/downloads/download.php?file=/technology/epp/downloads/release/juno/SR1/eclipse-jee-juno-SR1-win32-x86_64.zip}{64bit}\\
Mac OS X(Cocoa)&
\href{http://www.eclipse.org/downloads/download.php?file=/technology/epp/downloads/release/juno/SR1/eclipse-jee-juno-SR1-macosx-cocoa.tar.gz}{32bit}&
\href{http://www.eclipse.org/downloads/download.php?file=/technology/epp/downloads/release/juno/SR1/eclipse-jee-juno-SR1-macosx-cocoa-x86_64.tar.gz}{64bit}\\
Linux&
\href{http://www.eclipse.org/downloads/download.php?file=/technology/epp/downloads/release/juno/SR1/eclipse-jee-juno-SR1-linux-gtk.tar.gz}{32bit}&
\href{http://www.eclipse.org/downloads/download.php?file=/technology/epp/downloads/release/juno/SR1/eclipse-jee-juno-SR1-linux-gtk-x86_64.tar.gz}{64bit}\\
\end{tabular}
\end{center}

\subsection{Installation}
Pour installer Eclipse, il suffit de décompresser le fichier téléchargé. 

\bigskip

Lancer Eclipse :
\begin{itemize}
\item Sous windows cela crée un répertoire nommé "eclipse", exécuter le fichier eclipse.exe
\item Sous linux cela crée un répertoire nommé "/opt/eclipse", exécuter ./eclipse
\end{itemize}

Lors du premier lancement, une boite de dialogue vous demandera de sélectionner le répertoire racine de vos projets Eclipse. Soit vous sélectionnez celui proposé, auquel cas il sera créé ou, vous pouvez en choisir un autre.

\bigskip

Si vous utilisez Eclipse derrière un proxy, il vous faut ensuite configurer Eclipse pour le prendre en compte. Pour ce faire, accéder au menu \emph{Window/Preferences/Network~Connection}. Vous pouvez alors séléctionner \emph{Manual} comme \emph{Active Provider} et configurer votre proxy (pour l'IGN par exemple, il s'agit de \emph{proxy.ign.fr} avec le port \emph{3128}).

%-----------------------------------------------------------------------
\section{Installation de m2eclipse dans Eclipse}



